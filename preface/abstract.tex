\specialchapt{ABSTRACT}

Despite being an accepted and established forensic science practice, the process of matching bullets to determine whether they were fired from the same gun barrel has come under fairly intense scrutiny in recent years. This began in earnest in 2009 with a National Academy of Sciences report questioning the scientific validity of these methods. Further criticisms were made in a 2016 report by the President's Council of Advisors on Science and Technology (PCAST). After PCAST determined that there has only been one appropriately designed study to assess the accuracy of bullet matching methods, the report concluded that "[T]he current evidence falls short of the scientific criteria for foundational validity." The report also outlines a way forward by noting "A second—and more important—direction is . . . to convert firearms analysis from a subjective method to an objective method." This thesis attempts to take steps towards that goal. It begins by describing an automatic algorithm for matching bullet lands, and assesses this algorithm on the James Hamby study data. These ideas are then generalized in order to increase the prediction accuracy, determine operator effects in bullet scanning, handle the case of bullet land degradation, and apply to full bullet matches. Finally, a modern web-based database and software system for bullet matching is introduced, allowing for more seamless collaboration in the research community for assessing and improving these algorithms.
