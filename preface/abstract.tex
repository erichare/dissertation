\specialchapt{ABSTRACT}

Despite being an accepted and established forensic science practice, the process of matching bullets to determine whether they were fired from the same gun barrel has come under fairly intense scrutiny in recent years. This began in 2009 with a National Academy of Sciences report questioning the scientific validity of these methods. Further criticisms were made in a 2016 report by the President's Council of Advisors on Science and Technology (PCAST). After PCAST determined that there has only been one appropriately designed study to assess the accuracy of bullet matching methods, the report concluded "[...] the current evidence falls short of the scientific criteria for foundational validity". The report also outlines a way forward by noting "A second—and more important—direction is [...] to convert firearms analysis from a subjective method to an objective method.". This thesis attempts to take steps toward this goal. It begins by outlining principles of designing a modern software system that focuses on accessibility, modularity, extensibility, and reproducibility. These ideas are then used to formulate a bullet matching algorithm and software system, which allows for a quantitative assessment of the quality of a match using features from the literature, as well as new features which were deemed to be important. Finally, some steps that can be taken to expand and deploy this system into a real-world criminal justice scenario are provided.
